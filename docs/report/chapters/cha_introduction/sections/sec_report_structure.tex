%!TEX root = ../../../report.tex
\section{Report structure}
\label{sec:report_structure}
The present report is organized such that it starts with an overview of the evolution of legged locomotion and the current technologies applied to its study in \textbf{State of the art} [\ref{cha:state_of_the_art}].
It continues with an analysis of the robot solution and the definition of its design boundaries and criteria in the \textbf{Conception and initial analysis} chapter [\ref{cha:analysis}]
In the \textbf{Mathematical model of RuBi} [\ref{cha:mathematical_model}] a generalized kinematic and dynamic model of a biped is constructed, and the equations model of the application is sketched so that ca be used to size the actuators.
Then, in  the \textbf{Mechatronic design} chapter [\ref{cha:design}], divided into \textit{mechanics} [\ref{sec:mechanics}], \textit{Electronics} [\ref{sec:electronics}] and \textit{Software} [\ref{sec:software}], the integral and holistic design of the robot is carried out.
The \textbf{Simulation} chapter [\ref{cha:simulation}] is meant to justify the creation of the simulation tools included in the framework and explain their implementation.
It follows the \textbf{Construction process} chapter [\ref{cha:implementation}] where the processes of mechanical manufacturing, electronic wiring or the selection of the providers is explained.
In the \textbf{Experimental framework} chapter [\ref{cha:experiments}] the performed and the devised tests are presented, while in \textbf{Economical aspects} [\ref{cha:economical_aspects}] the budget analysis of the project is broken down.
The report finishes with the chapters \textbf{Results} [\ref{cha:results}], \textbf{Discussion} [\ref{cha:discussion}] and \textbf{Conclusions and further work} [\ref{cha:conclusions}].