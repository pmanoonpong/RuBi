%!TEX root = ../../../report.tex
\section{Report structure}
\label{sec:report_structure}
The document is organized so it starts with an intensive analysis of the current technologies and solutions in \textbf{State of the art} [\ref{cha:state_of_the_art}]; continues with an analysis of the problem and design boundaries are established in the \textbf{Analysis} chapter [\ref{cha:analysis}]; in the \textbf{Mathematical model of the robot} [\ref{cha:mathematical_model}] a generalized kinematic and dynamic model of a biped is got that is then used to size the actuators; then, in  the \textbf{Design} chapter [\ref{cha:design}], divided in \textit{mechanics} [\ref{sec:mechanics}], \textit{Electronics} [\ref{sec:electronics}] and \textit{Software} [\ref{sec:software}], the integral and holistic design of the robot is carried out; the \textbf{Simulation} chapter [\ref{cha:simulation}] is meant to explain why and how the simulation tools where included in the framework; it follows with an \textbf{Implementation} chapter [\ref{cha:implementation}] where processes as the manufacturing, the wiring or the election of the providers is explained; in the \textbf{Experiments} chapter [\ref{cha:experiments}] are depicted the performed tests, whilst in \textbf{Economical aspects} [\ref{cha:economical_aspects}] the budget of the project is specified; it finishes with the chapters \textbf{Results} [\ref{cha:results}], \textbf{Discussion} [\ref{cha:discussion}] and \textbf{Conclusions} [\ref{cha:conclusions}].