%!TEX root = ../../../report.tex

\section{Geometrical dimensions of the frame} % (fold)
\label{sec:dimensions}
The selection of the final dimensions of RuBi corresponds to an iterative process in which power requirements and achieving human-like kinematics have been the main constraints when assessing the size of the robot.
A first approach was taken from \cite{grimmer}, Manuscript I, where normalized values of power required per joint and per kilogram of the structure can be found.
The first iteration targeted a robot of dimensions $m = 1Kg$ and length from the hip joint to the tip toes of $L = 0.6 m$ for a fully stretched leg.
The final parameters from the last iteration can be found in \ref{tab:limb_physical_properties}.

Once estimated an overall length of the structure, an initial value of the dimensions of each link were decided to be obtained based on the German section of the ISO 7250-2 \cite{iso_measurements} and the DIN 33402-2 \cite{din_measurements1} norms, following the idea of mimicking human-like motion as closely as possible.
These two norms are standard references in industry and were considered a general enough source of information.
Figure \ref{fig:human_measurements} depicts some of the human dimensions extracted from the norms and applied to the design of RuBi.
Their names and values for male between 18-65 year old and in percentile 50 can be found in table \ref{tab:din_proportions}.

\begin{figure}[htb]
	\centering
	\begin{subfigure}[b]{0.3\textwidth}
        \includegraphics[width=\textwidth]{figures/din_measurements.pdf}
        \caption{Left foot}
        \label{fig:din1}
    \end{subfigure}
    \begin{subfigure}[b]{0.4\textwidth}
        \includegraphics[width=\textwidth]{figures/din_measurements2.pdf}
        \caption{Hip}
        \label{fig:din2}
    \end{subfigure}
	\caption{Lower body measurements used for RuBi. Picture adapted from \cite{din_measurements1}.}
	\label{fig:human_measurements}
\end{figure}


\begin{table}
\begin{center}
	\begin{tabular}{c | c | c | c}
	  Index & Definition & Value & \% of Stature \\
	  \hline
	  A & Stature (body height) & ** & 100 \\
	  B & Crotch height & ** & ** \\
	  C & Femur height & ** & ** \\
	  D & Tibial height & Not in norm & **\\
	  E & Ankle-toe tip distance & Not in norm & ** \\
	  F & Buttocks-leg length & ** & ** \\
	  G & Sole length & ** & **
	\end{tabular}
	\caption{Human proportions from DIN 33402-2.}
	\label{tab:din_proportions}
\end{center}
\end{table}

\subsection{Limbs length} % (fold)
\label{sub:limbs_lengths}
The dimensions needed to create a simplified kinematic model of one leg are the limbs lengths, which are the straight line distances measured from two consecutive joints.
They have been called $l_{i}$ where $i$ is the robot link + joint as per table \ref{tab:limb_index} and can be seen in Figure \ref{fig:kinematics}.
However, the norm does not determine all of them.
\begin{table}
\begin{center}
	\begin{tabular}{c | c | c}
	  $l_{i}$ & Limb \\
	  \hline
	  $l_{1}$ & Hip + thigh & C \\
	  $l_{2}$ & Knee + Foreleg & D\\
	  $l_{3}$ & Ankle + foot & E 
	\end{tabular}
	\caption{Limbs index.}
	\label{tab:limb_index}
\end{center}
\end{table}
Since $C$ and $E$ are not standard measurements in industry, they have been obtained as follows:

\paragraph{The Femur height}
The crotch height, denoted as $B$ in the figure, has been averaged with the buttocks-leg length and the tibial height + an empirical approximation of the height of the ankle have been subtracted from the result, obtaining $C$.

\paragraph{The ankle-toe tip distance}
Since this distance is not standard either, it has been obtained once again adjusting the sole length with empirical measurements.
% subsection limbs_lengths (end)

\subsection{Hip and sole width} % (fold)
\label{sub:subsection_name}
The hip width sitting, as defined in the norm and for the same sample group than before, has been used as starting point to define the final width of the structure of the hip.
This standard measurement has been scaled down to a human of length L from hip joint to toe tip, which would correspond to $L=C+D+E$, using the proportion between stature and lower body lengths, also obtained from the norm.
The same process has been conducted to calculate the implemented width of the foot sole.

% subsection hip_sole_width (end)

No other standard measurement has been used as reference for the design, such as thigh or lower leg circumferences, since they do not affect the kinematics of the structure although they have a big influence in its dynamics.
The criteria followed to model the dynamics of the robot is explained in \ref{sec:physical_properties}.


%% Here we just describe the process followed to obtain the final dimensions. 
%% The final results are shown in Chapter Results --> add tables and Froude number calculus


% section dimensions (end)