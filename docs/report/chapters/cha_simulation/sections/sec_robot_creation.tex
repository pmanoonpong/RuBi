%!TEX root = ../../../report.tex
\section{Robot model creation} % (fold)
\label{sec:robot_creation}
One of the goals of the framework is to facilitate its further use and development.
For this reason this section is dedicated to explain how the robot model was created for Gazebo, so that it can be modified in the future.

A robot is defined as a set of links joined, where a link contains at least three blocks of information:
\begin{enumerate}
   \item The collision model: used to calculate the collisions with other agents in the simulation.
   \item The visual model: uniquely used for visual purposes.
   \item The inertial information: defines physical properties like the mass and the moments of inertia of the link.
\end{enumerate} 

A trade-off must be found for the precision of the collision model between accuracy and speed.
The more detailed is the model, the more accurate its simulation will become, but this implies a higher computational load and processing time.
The general advice is to have two simulation models: one for visual purposes and the other one for collisions.
In the case of the frame of RuBi the visual models are obtained directly by exporting the parts from SolidWorks, while the collision models are made out of primitive gazebo shapes (cubes, cylinders, spheres...) for the whole body except the feet.
Since the robot is not meant to collide with obstacles due to its nature and setup, the possible collisions in the body are not a priority, then each link is represented as a cylinder of the same length of the CAD model and of a radius enough to cover the whole part.
An example can be seen in the Figure \ref{fig:collision_model}.

The moments of inertia of each individual link are obtained directly from SolidWorks.
The CAD model includes the materials and thus the masses.
From these, the program is able to calculate the moments of inertia.
The Figure \ref{fig:moments_of_inertia} shows the representation of these parameters in the simulation.


\begin{figure}[hb!]
  \centering
  \begin{subfigure}{.45\textwidth}
    \centering
    \includegraphics[width=.5\linewidth]{figures/gazebo_collision_model.png}
    \caption{Gazebo showing the visual model, the moments of inertia and the collision model.}
    \label{fig:collision_model}
  \end{subfigure}
  \begin{subfigure}{.45\textwidth}
    \centering
    \includegraphics[width=.5\linewidth]{figures/gazebo_intertia_moments.png}
    \caption{Gazebo showing the visual model and the moments of inertia.}
    \label{fig:moments_of_inertia}
  \end{subfigure}
  \caption{Gazebo showing physical and visual propperties of RuBi.}
\end{figure}

It is essential to make sure that the visual models and moments of inertia are exported referred to the correct coordinate systems.
In the case of the ankle, for example, the STL file and the moments of inertia are calculated from the coordinate system whose origin is positioned in the middle of the rotational axis of the joint, with its Z axis attached to the rotational axis and Y pointing to the biggest extension of the part.
This can be seen in the Figure \ref{fig:solidworks_ankle_coodinate_system}.

\begin{figure}[ht!]
  \centering
  \includegraphics[width=0.75\linewidth]{figures/solidworks_ankle_coordinate_system}
  \caption{Coordinate system of the left ankle in SolidWorks.}
  \label{fig:solidworks_ankle_coodinate_system}
\end{figure}

Finally, in the Xacro file the information that lacks the URDF must be added in order to simulate with Gazebo.
To do so, the following has to be added to the RuBi description:
\begin{enumerate}
  \item \textbf{Plug-in for ROS Control}: Offers an interface for using ROS Control inside Gazebo.
  \item \textbf{Contact sensors}: Simulates contact sensors as in the feet of the real robot.
  \item \textbf{Friction coefficients}: Defines the friction between the feet and the floor.
\end{enumerate}

The robot has been simulated with direct drive transmission, since the springs configurations has not been implemented because of the lack of time.
This is left as further work.

% section robot_creation (end)