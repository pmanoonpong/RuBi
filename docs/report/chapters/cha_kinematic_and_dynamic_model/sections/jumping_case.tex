\section{The jumping case} % (fold)
\label{sec:jumping_case}
Modeling the dynamics of walking gaits in a bipedal robot with 6 DOF has proved significantly complex and time consuming, even for the two-dimensional case.
However, it is necessary to calculate the most extreme torque and velocity values to apply to every joint during the motion and thus gauge the required motors for the application.
In order to ease the task, instead of calculating the motion equations for the walking or running cases, it was decided to model the physics of the static jumping case.
This was decided under the assumption that a structure able to perform a static, vertical jump of determined characteristics on one leg could be able to fulfill the power requirements necessary for running. Both in from an actuation and structural point of view.
% section The_jumping_case (end)