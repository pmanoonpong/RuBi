%!TEX root = ../../../report.tex

\section{The jumping case} % (fold)
\label{sec:jumping_case}
Modeling the dynamics of walking gaits in a bipedal robot with 6 DOF has proved significantly complex and time consuming, even for the two-dimensional case.
However, it is necessary in order to calculate the most extreme torque and velocity values to apply to every joint during the motion and thus gauge the required motors for the application.
In order to ease the task, instead of calculating the motion equations for the walking or running cases, it was resolved to model the dynamics of the jump case.
This was decided under the assumption that a structure able to perform a static, vertical jump of determined characteristics on one leg could be able to fulfill the power requirements necessary for running, both from an actuation and a structural point of view.
This assumption was built upon the findings introduced in \cite{jump-run1} and \cite{jump-run2}.

\subsubsection{Vertical jump dynamics} % (fold)
\label{ssub:static_jumping_dynamics}
As explained in section \ref{sec:dimensions}, the dimensions of the robot and its parameters have been conceived to imitate human characteristics and capabilities.
Therefore the jump analysis here will be carried out in resemblance of the human case, based on \cite{jump-dynamics1} and aiming at applying the results to a robot platform, as in \cite{jump-dynamics2}.
The jump case analyzed here can be described according to \cite{jump-dynamics1} as a static, squat jump over one foot with take-off and without counter-movement (SJ-NAS).
The whole jump cycle has been divided into two phases with different analysis: an impulse and a flight period.

\paragraph{The impulse phase}
From the initial position at the beginning of the jump to the instant when the toes of the standing foot take off the ground and the ground reaction force ceases.
The change in momentum caused by the application of the a vertical force with the feet to the ground F during a period t is given in equation

\begin{equation}
	F  t = m  \Delta V	
\end{equation} 


% subsubsection static_jumping_dynamics (end)

% section The_jumping_case (end)