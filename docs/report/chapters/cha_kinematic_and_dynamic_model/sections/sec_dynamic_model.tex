%!TEX root= ../../../report.tex
\section{Dynamic model}
\label{sec_dynamic_model}
The goal of this step is to calculate the relationship between an external force applied to the toe (ground reaction force while jumping) and the necessary torques in the joints for dealing with that external disruption.
This relation can be expressed as a set of second order differential equations represented for the general case as in equation \ref{eq:dynamics_eq1}.

\begin{equation}
	\label{eq:dynamics_eq1}
	B(q)\ddot{q} + C(q,\dot{q}) + g(q) = \tau
\end{equation}

Where $B(q)$ is the inertia matrix, $c(q,\dot{q})$ contains the centrifugal and coriolis acceleration terms and $g(q)$ represents gravity, as in \cite{dynamics1} and \cite{dynamics2}. 
For an open kinematic chain as this, three methodologies to obtain the above equation where object of study:

\begin{itemize}
	\item A simplified Euler-Lagrange algorithm, as introduced in \cite{E-L1}, which makes use of the Lagrangian formulation to describe the behavior of the system through work and energy.
	\item The so called Energy Method, presented in \cite{asada} and consisting in finding the relation between the force in the end-effector and the joint torques through the Jacobian of the kinematic chain.
	\item The Newton-Euler algorithm, through which the dynamics of the system can be expressed in terms of forces and moments applied in each member of the chain.
\end{itemize}

It was finally decided to apply the third option due to the fact that it was faster to implement than the E-L algorithm and more reliable than the Energy method.
The recursive Newton-Euler algorithm is founded in classical mechanics and its a recursive method that computes in two steps the velocities and accelerations of every component of a kinematic chain and their forces and torques on the joints.
In \ref{eq:N-E_eq1}, the equations of motion for an individual link are shown.

\begin{equation}
\label{eq:N-E_eq1}
	\begin{aligned}
		f_{i-1, i} - f_{i,i+1} + m_{i} g - m{i} \dot{V}_{ci} &= 0 \\
		N_{i - 1 , i} - N_{i , i + 1} - ( r_{i - 1 , i} + r_{i , Ci} ) \times f_{i - 1 , i} + ( - r_{i , Ci} ) \times ( - f_{i , i + 1}) - \dot{I}_{i} \omega_{i} - \omega_{i} \times ( I_{i} \omega_{i} ) &= 0 \\
	\end{aligned}
\end{equation}

These equations contain the coupling forces and moments applied to the link by the immediate ones, however, they cannot be used with them. 
The closed-form equations must be derived from them so that they can be applied to the algorithm.
This is done by substituting the coupling forces in the final set of 6 equations obtained for $i=3$ and substituting 

\begin{equation}
\label{eq:torques}
	N_{i - 1 , i} = \tau_{i}
\end{equation}
