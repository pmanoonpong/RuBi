%!TEX root = ../../report.tex
\chapter{Mathematical model of the robot} % (fold)
\label{cha:mathematical_model}

At first, the motivation to start the development of the mathematical model was the seek of a set of equations with which calculating the power required for the actuators could be easily achievable.
The main goal was to compute the necessary data to select joint actuators able to keep up with the requirements of the application.
However, by definition to do so the application needed to be defined to the extent that these dimensions could be obtained.
This chapter contains the definition of the task the final robot would be subject to, together with and explanation of the assumptions made in order to simplify the resulting model. 
After, the necessary kinematic model of the robot is presented as an intermediary step to compute the simplified dynamic model of the robot, which is introduced next to it.
All the algorithms described in this chapter have been implemented in MATLAB and can be found in appendix. %\ref{}.

%!TEX root = ../../../report.tex

\section{The jumping case} % (fold)
\label{sec:jumping_case}
Modeling the dynamics of walking gaits in a bipedal robot with 6 DOF has proved significantly complex and time consuming, even for the two-dimensional case.
However, it is necessary in order to calculate the most extreme torque and velocity values to apply to every joint during the motion and thus gauge the required motors for the application.
In order to ease the task, instead of calculating the motion equations for the walking or running cases, it was resolved to model the dynamics of the jump case.
This was decided under the assumption that a structure able to perform a static, vertical jump of determined characteristics on one leg could be able to fulfill the power requirements necessary for running, both from an actuation and a structural point of view.
This assumption was built upon the findings introduced in \cite{jump-run1} and \cite{jump-run2}.

\subsubsection{Vertical jump dynamics} % (fold)
\label{ssub:static_jumping_dynamics}
As explained in section \ref{sec:dimensions}, the dimensions of the robot and its parameters have been conceived to imitate human characteristics and capabilities.
Therefore the jump analysis here will be carried out in resemblance of the human case, based on \cite{jump-dynamics1} and aiming at applying the results to a robot platform, as in \cite{jump-dynamics2}.
The jump case analyzed here can be described according to \cite{jump-dynamics1} as a static, squat jump over one foot with take-off and without counter-movement (SJ-NAS).
The whole jump cycle has been divided into two phases with different analysis: an impulse and a flight period.

\paragraph{The impulse phase}
From the initial position at the beginning of the jump to the instant when the toes of the standing foot take off the ground and the ground reaction force ceases.
The change in momentum caused by the application of the a vertical force with the feet to the ground F during a period t is given in equation

\begin{equation}
	F  t = m  \Delta V	
\end{equation} 


% subsubsection static_jumping_dynamics (end)

% section The_jumping_case (end)
\section{Kinematic model}
\label{sec_kinematic_model}
%!TEX root= ../../../report.tex
\section{Dynamic model}
\label{sec_dynamic_model}
The goal of this step is to calculate the relationship between an external force applied to the toe (ground reaction force while jumping) and the necessary torques in the joints for dealing with that external disruption.
This relation can be expressed as a set of second order differential equations represented for the general case as in equation \ref{eq:dynamics_eq1}. 
\begin{equation}
	\label{eq:dynamics_eq1}
	B(q)\ddot{q} + C(q,\dot{q})\dot{q} + g(q) = \tau
\end{equation}
Where $B(q)$ is the inertia matrix, $c(q,\dot{q})$ contains the centrifugal and coriolis acceleration terms and $g(q)$ represents gravity, as in \cite{dynamics1} and \cite{dynamics2}. 
It must be remarked here that the constructed model does not introduce joint friction terms.
For an open kinematic chain as this, three methodologies to obtain the above equation where object of study:
\begin{itemize}
	\item A simplified Euler-Lagrange algorithm, as introduced in \cite{E-L1}, which makes use of the Lagrangian formulation to describe the behavior of the system through work and energy.
	\item The so called Energy Method, presented in \cite{asada} and consisting in finding the relation between the force in the end-effector and the joint torques through the Jacobian of the kinematic chain.
	\item The Newton-Euler algorithm, through which the dynamics of the system can be expressed in terms of forces and moments applied in each member of the chain.
\end{itemize}
It was finally decided to apply the third option due to the fact that it was faster to implement than the E-L algorithm and more reliable than the Energy method.
The Newton-Euler algorithm is founded in classical mechanics and its a recursive method that computes in two steps the velocities and accelerations of every component of a kinematic chain and their forces and torques on the joints.
In \ref{eq:N-E_eq1}, the equations of motion for an individual link are shown.
\begin{equation}
\label{eq:N-E_eq1}
	\begin{aligned}
		f_{i-1, i} - f_{i,i+1} + m_{i} g - m{i} \dot{V}_{ci} =& 0 \\
		N_{i - 1 , i} - N_{i , i + 1} - ( r_{i - 1 , i} + r_{i , Ci} ) \times f_{i - 1 , i} + ( - r_{i , Ci} ) \times ( - f_{i , i + 1}) - I_{i} \dot{\omega_{i}} - \omega_{i} \times ( I_{i} \omega_{i} ) =& 0 \\
		i = 1,..., N 
	\end{aligned}
\end{equation}
These equations contain the coupling forces and moments applied to the link by the immediate ones, however, they cannot be used with them. 
$\omega_{i}$ and $\dot{\omega_{i}}$ represent respectively the angular velocity and acceleration vectors of link $i$ and can be obtained from the joint velocities and accelerations, computed in the previous section, as in equation \ref{eq:angular_magnitudes}.
$I_{i}$ are the inertia matrices of the links, calculated in SolidWorks for the final design of RuBi.
\begin{equation}
\label{eq:angular_magnitudes}	
	\begin{aligned}
		\omega_{i} &= \omega_{i-1} + \xi_{i}\dot{q}_{i}Z_{base}^i\\
		\dot{\omega_{i}} &= \dot{\omega_{i-1}} + \xi_{i}[\ddot{q}_{i}Z_{base}^i + \dot{q}_{i}\omega_{i-1} \times Z_{base}^i]
	\end{aligned}
\end{equation}
The closed-form equations must be derived from them so that they can be applied to the algorithm.
This is done by substituting the coupling forces in the final set of 6 equations obtained for $N=3$ and substituting \ref{eq:torques} in the resulting equations.
\begin{equation}
\label{eq:torques}
	N_{i - 1 , i} = \tau_{i}
\end{equation}
Equation \ref{eq:torques} is valid only for the planar case.
After these steps, a set of three equations of the form \ref{eq:dynamics_eq1} is obtained. 
As parameters, they contain the masses of the links, their vectorial positions, the gravity term and their inertia matrices. 

This set of equations calculates the necessary torques on the joints as a function of the joints displacements and the external forces and torques applied on the system, as represented in \ref{eq:tau_q}.
\begin{equation}
\label{eq:tau_q}
	\tau(t)_{i} = f(q_{i}(t), \dot{q}_{i}(t), \ddot{q}_{i}(t), \tau_{ext}, F_{ext})
\end{equation}
However, in order to use it the functions that model the trajectories of joints in the joint space must be approximated.
For the ideal static, vertical jump case under study here, the movement of the toe has been constrained to a vertical displacement along the $Y$ axis in order to simplify this task.
Thus, the trajectory of the end-effector of the kinematic chain, $P_{3}$ in Figure \ref{fig:f-t}, can be easily approximated as a function of the form shown in 
\begin{equation}
\label{eq:toe_trajectory}
	\begin{aligned}
	x_{3}(t) &= 0 \\
	y_{3}(t) &= y_{3}(t_{o}) + \cfrac{y_{3}(t_{f}) - y_{3}(t_{o})}{(t_{f} - t_{o}) (t - t_{o})} \\
    \theta(t) &= \theta(t_{o}) + \cfrac{\theta(t_{f}) - \theta(t_{o})}{(t_{f} - t_{o}) (t - t_{o})} 
    \end{aligned}
\end{equation}
Therefore, if equation \ref{eq:toe_trajectory} is used as the input to the inverse kinematic model, the joint displacements will become dependent on the linear displacement of $P_{3}$, and equations \ref{eq:tau_q} will be rewritten as \ref{eq:tau_p}.
\begin{equation}
\label{eq:tau_p}
	\tau(t)_{i} = f(P_{3}(t), \tau_{ext}, F_{ext})
\end{equation}
\section{Actuators}
\label{sec_actuators}
%!TEX root= ../../../report.tex

\section{Springs}
\label{sec_springs}
%Furthermore, all the studies conducted in the influence of compliance in legged locomotion have been carried out over empirical data recorded for the specific task under analysis.
%All the studies found to optimize the value of springs focus only in a specific velocity and pattern --> we want a wide range of speed and both walk and run

%!TEX root= ../../../report.tex

\section{Model-based controllers}
\label{sec_dynamic_controller}
As introduced in the previous sections, the equations of motion derived for RuBi can be used to compute the necessary output values of its actuators in order to perform an input toes trajectory and external forces, as expressed in \ref{eq:tau_p}.
By definition, this could be sufficient to, appropriately used, be utilized as the base of a controller for the robot, without accounting the own dynamics of the actuators.
As an example of the above said, and despite the early stage in its development in which this model is left, some data has been computed and plotted for proving the concept.
The only trajectory planner current implemented computes vertical straight paths for the toes.
However, more complex trajectory equations could be used to achieve different locomotion patterns.

\begin{figure}[htb]
	\centering
	\includegraphics[width=0.5\textwidth]{figures/kinematics_sim.pdf}
	\caption{Joints trajectories from kinematics model.}
	\label{fig:controller_position}
\end{figure}

An initial and final toes position has been inputed to \ref{eq:toe_trajectory} to define an example trajectory for the toes in one leg, together with a random value for $\Delta h$.
The output of the kinematics computed for $i=1,...,N$ steps is shown in Figure \ref{fig:controller_position}, where it has been rearranged for visualization.
It can be seen that the toes trajectory does not cover the full range of a vertical jump.

\begin{equation}
\label{eq:joint_vel}
	\dot{\theta}_{i,j} =\frac{ \abs{ \theta_{j}(t_{i}) - \theta_{j}(t_{i-1}) } }{ t_{step} }
\end{equation}

For the trajectory described and $\Delta h$, the impulse equation in \ref{eq:impulse} has been computed and the values $(F_{min}, t_{max})$ from \ref{eq:work} have been used as input to the dynamic model.
The result of computing \ref{eq:dynamics_eq1} for the same time steps is shown in \ref{fig:controller_torque}, together with the solution for equation \ref{eq:joint_vel} in \ref{fig:controller_speed}, calculated in an analogue way.

\begin{figure}[htb]
    \centering
    \begin{subfigure}{0.49\textwidth}
        \includegraphics[width=\textwidth]{figures/torque-time.pdf}
		\caption{Joints torque as a function of time.}
		\label{fig:controller_torque}
	\end{subfigure}	
    \begin{subfigure}{0.49\textwidth}
        \includegraphics[width=\textwidth]{figures/speed-time.pdf}
		\caption{Joints velocity as a function of time.}
		\label{fig:controller_speed}
    \end{subfigure}
    \caption{Joint torque and velocity as a function of time.}
\end{figure}

The ranges in which the results in \ref{fig:controller_torque} and \ref{fig:controller_speed} are found seem consistent with the obtained during the experiments detailed in \ref{sub:suitability_of_the_motor_model_for_the_application}.
However, the analysis of their accuracy would entail a more precise construction of the mathematical model and study of kinematic chains dynamics which laid out of the scope of this project since it was not considered a priority goal.


% chapter kinematic_and_dynamic_model (end)