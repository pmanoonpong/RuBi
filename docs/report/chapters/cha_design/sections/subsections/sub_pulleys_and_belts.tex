%!TEX root = ../../../../report.tex
\subsection{Pulleys and belts} % (fold)

\label{sub:pulleys_and_belts}
Section \ref{sec:joints} has been dedicated to analyze and define the kind of motor configuration and transmission system which is going to be used for each joint.
In the case of the knee and the ankle, the combination \textit{motor + gearbox + belt and pulleys} was selected, as explained before, following inertial-reduction criteria.
The transmission system designed to convey the motion from the motor shaft to the joint consists finally in the \textit{pulley+belt} presented here plus a torsional spring in series used to transfer the rotation movement from the joint pulley to the limb, as discussed in \ref{sub:compliance}.
As a result, an interface between the pulley and a spring holder was needed.
Hence, the design of the pulley itself has been studied in terms of two factors: \textit{precision and backlash reduction} and \textit{integration with the series torsional spring}.

\subsubsection{System backlash reduction} % (fold)
\label{ssub:precision_and_backlash_reduction}
All the pairs of pulleys are meant to have the same diameters since they are not planned to be used for torque-speed ratio modifications between actuators and joints. 
Thus, the final value of this dimension obeys only to other components dimensions constraints.
With that degree of freedom, the aim of the design process is to optimize the pulley in order to minimize their associated backlash, defined as the clearance between timing belt teeth and timing belt pulley grooves.
Despite the fact that the platform is going to be used mainly with adaptive controllers (e.g. ANN-based) and therefore the mechanical optimization is not a priority, the reduction of mechanical uncertainties is always sought.

After analyzing the current solutions in the market, several non-backlash solutions were found.
The optimal one seemed to be the Gates GT3 Synchronous Belts\footnote{http://www.gates.com/products/industrial/industrial-belts/synchronous-belts/powergrip-gt3-belts}, able to fulfill the requirements of the presented application.
The withdrawals of this option were the time constraints for ordering such parts and the increase in the final price of the product. 
But also, the integration with the series torsional spring.
% subsubsection precision_and_backlash_reduction (end)

\subsubsection{Integration of the pulleys with the series torsional springs} % (fold)
\label{ssub:integration_with_the_series_rotational}
An alternative solution was to design and manufacture the pulley, which would allow a complete control of the design and manufacturing process, thus yielding the possibility of integrating in a single part the pulley and the spring holder.
At first, the GT3 design from Gates was intended to be utilized as a model.
However its design, which is described in U.S. Patent Number 4,515,577, is patented and not open to the public.
As an alternative, the ISO 13050:2014 \cite{ISO13050}, following the type T, was used to create the model of the customized pulley.
This choice was based on its focus in efficiency and reduction of backlash, together with its optimality for accurate movements with high torques and low speeds, as the intended application here.
The physical properties of the pulleys as the number of teeth, width, etc... were selected according to the ISO 5295:1987 \cite{ISO5295}, and as mentioned before, with the view on their manufacturability.

The interface between the described pulley and the limb was created as a torsional spring leg holder attached to the pulley's side.
An equivalent piece was placed on the joint, turned $90\degree$ for holding the other leg of the spring, which is meant to be coiled around the axis of the joint.
Based on the two ISO norms cited before and after some iterations based on experimental tests, the pulley T2,5 of 19 teeth gave the expected behavior.
In Figure \ref{fig:motor_pulley} a detail of the final pulleys can be seen.
Figure \ref{fig:serial_spring_pulley} shows the pulley side of the series spring interface.
% subsubsection integration_with_the_series_rotational (end)

\subsubsection{Belts selection} % (fold)
\label{ssub:belts}
From \ref{ssub:integration_with_the_series_rotational} a system pulleys-belt using T2,5 teeth profile was selected.
For the implementation of the belt system, an open belt whose initial tension can be adjusted with a zip-tie is chosen.
This allows to adjust the tension at any moment without the need of further tools.
Besides, it facilitates the system maintenance.
However, no studies were conducted on the right amount of tension to apply to the belts. 
The complex dependency of the tension values on the application requirements led to their empirical adjustment as the most suitable option.
% subsubsection belts (end)

% subsection pulleys_and_belts (end)