%!TEX root = ../../../../report.tex
\subsection{Computer-Aided Design (CAD)} % (fold)
\label{sub:computer_aided_design}
Based on the defined boundaries of the previous sections, the necessary parts have been designed.
The modeling has been design-oriented manufacturing having in mind the 3D printing technology where not all kind of geometries can be printed with the same resolution and quality.

Generally, the parts have followed a round design avoiding cutting edges except where completely necessary.
Some corrections have been applied to the tolerances of the internal holes as explained in the section \ref{sub:arc_compensation}.
The implementation of the CAD design can be considered partially parametric due to, for its construction, a file with some parameters is used in several parts of the design.
This works so in the reconstruction of the part the program reads the file and change everything according to its parameters.
The used values are shown in the appendix \ref{app:cad_parameters}.
The results can be seen in the figures from \ref{fig:left_foot} to \ref{fig:knee_upper}.

\begin{figure}[ht!]
    \centering
    \begin{subfigure}[b]{0.49\textwidth}
        \includegraphics[width=\textwidth]{figures/legs_foot.jpg}
        \caption{Left foot}
        \label{fig:left_foot}
    \end{subfigure}
    \begin{subfigure}[b]{0.49\textwidth}
        \includegraphics[width=\textwidth]{figures/legs_hip.jpg}
        \caption{Hip}
        \label{fig:hip}
    \end{subfigure}
\end{figure}    

\begin{figure}[ht!]
    \ContinuedFloat % continue from previous page
    \begin{subfigure}[b]{0.49\textwidth}
        \includegraphics[width=\textwidth]{figures/legs_parts.jpg}
        \caption{Additional designed parts}
        \label{fig:mouse}
    \end{subfigure}
    \begin{subfigure}[b]{0.49\textwidth}
        \includegraphics[width=\textwidth]{figures/legs_pulley.jpg}
        \caption{Left ankle serial spring pulley}
        \label{fig:serial_spring_pulley}
    \end{subfigure}
\end{figure}    

\begin{figure}[ht!]
    \ContinuedFloat % continue from previous page
    \begin{subfigure}[b]{0.49\textwidth}
        \includegraphics[width=\textwidth]{figures/legs_pulley_motor.jpg}
        \caption{Motor pulley}
        \label{fig:motor_pulley}
    \end{subfigure}
    \begin{subfigure}[b]{0.49\textwidth}
        \includegraphics[width=\textwidth]{figures/legs_knee_lower.jpg}
        \caption{Left lower knee}
        \label{fig:lower_knee}
    \end{subfigure}
\end{figure}

\begin{figure}[ht!]
    \ContinuedFloat % continue from previous page
    \begin{subfigure}[b]{0.49\textwidth}
        \includegraphics[width=\textwidth]{figures/legs_ankle_upper.jpg}
        \caption{Left upper ankle}
        \label{fig:ankle_upper}
    \end{subfigure}
    \begin{subfigure}[b]{0.49\textwidth}
        \includegraphics[width=\textwidth]{figures/legs_hip_lower.jpg}
        \caption{Left lower hip}
        \label{fig:hip_lower}
    \end{subfigure}
\end{figure}

\begin{figure}[ht!]
    \ContinuedFloat % continue from previous page
    \begin{subfigure}[b]{0.49\textwidth}
        \includegraphics[width=\textwidth]{figures/legs_hip_pinion.jpg}
        \caption{Hip's pinion}
        \label{fig:hip_pinion}
    \end{subfigure}
    \begin{subfigure}[b]{0.49\textwidth}
        \includegraphics[width=\textwidth]{figures/legs_knee_upper.jpg}
        \caption{Left upper knee}
        \label{fig:knee_upper}
    \end{subfigure}
\end{figure}

% subsection computer_aided_design (end)