%!TEX root = ../../../../report.tex
\subsection{Limb profile} % (fold)
\label{sub:limb_profile}
Based on the requirements of weight and its distribution defined in the analysis of the joints \ref{sec:joints}, the links have been decided to have in the upper extreme the motors. 
This leads to use a transmission system, such as belt and pulleys, which leaves for the rest of link an structural function that can also adopt the task of wiring placement.
Thus, a light weight section that satisfy the conditions of deformation and stress maximum will be chosen.
Carbon fiber is an ideal material to achieve this conditions of weight and stress so an quantitative analysis has been made calculating the optimal solution and then rounding it for all the possible profiles offered by the given provider.
The provider was chosen due to the previous experiences that the Mærsk Mc-Kinney Møller Institute had with carbon fiber orders.

The section profile offered \footnote{http://www.easycomposites.co.uk/\#!/cured-carbon-fibre-products/} are: \textit{Rod}, \textit{Tube}, \textit{Box}. The \textit{Stripe} and the \textit{Angle} are discarded due to its asymmetrical geometry that will will lead to less predictable scenarios.
The study case is show in the figure PUT A FUCKING PICTURE IN HERE JORGE, where the impact force can be decomposed in a pure bending effort and a pure compression.
Due to the resistance of the carbon fiber in pure compression is bigger than in bending, only this last case has been studied.

\subsubsection{Torque calculation} % (fold)
\label{ssub:torque_calculation}
  For both cases, the profile of the lower limb and the upper, the torque generated by the impact force calculated in the section \ref{sub:impact_force}, is calculated.
  This torque is different in the lower and the upper due to the upper takes the distance from the foot to the hip while the other one is only from the foot to the knee:
  \begin{equation}
  \begin{aligned}
     M_{lower\ limb} = F \cdot d = 294.4 \cdot 0.2126 = 62.57 \\
     M_{upper\ limb} = F \cdot d = 294.4 \cdot (0.2126 + 0.2622) = 139.73
  \end{aligned}
  \end{equation}
% subsubsection torque_calculation (end)

\subsubsection{Profile study} % (fold)
\label{ssub:profile_study}
  The bending effort causes two sort of problems: (1) the possible break in the supporting point and (2) the deformation suffered by the beam.
  The break will occur when the tensions created will be over the ultimate tension in compression or tension of the selected material.
  This comes defined in the equation \ref{eq:tension} for simetric sections and when an only torque is being applied.
  \begin{equation}
  \label{eq:tension}
    \sigma _{compression} = \sigma _{tension} = \frac{M h_{CG}}{I_x}
  \end{equation}

  Meanwhile the deformation in the extreme can be calculated with the equation \ref{eq:deformation} if the case is simplified to the occured in the figure PUT A FUCCKING PICTURE.
  This is, when the legs is completely streched, that will cause the biggest stresses.

  \noindent\begin{minipage}{0.2\textwidth}% adapt widths of minipages to your needs
    \includegraphics[width=\linewidth]{figures/sdu_logo.pdf}
  \end{minipage}%
  \hfill%
  \begin{minipage}{0.8\textwidth}
    \begin{equation}
    \label{eq:deformation}
      y_L = \frac{P z^2}{6EI}(3L-z) = \frac{P L^2}{6EI}(2L) = \frac{P L^2}{3EI}
    \end{equation}
  \end{minipage}


  For the selected profiles the equations that define the $\sigma _{compression}$ are shown:


  \noindent\begin{minipage}{0.2\textwidth}% adapt widths of minipages to your needs
      \includegraphics[width=\linewidth]{figures/sdu_logo.pdf}
  \end{minipage}%
  \hfill%
  \begin{minipage}{0.8\textwidth}
    \begin{equation}
      \begin{align*}
      \sigma _{compression} = \sigma _{tension} &= \frac{M h_{CG}}{I_x} = \frac{4 r_2}{\pi(r_2 ^4 - r_1 ^4)} M \\
      y_L &= \frac{P L^2}{3EI} = \frac{P L^2}{3EI} \\
      h_{CG} &= r_2 \\
      I_x = I_y &= \frac{\pi}{4} (r_2 ^4 - r_1 ^4)
      \end{align*}
    \end{equation}
  \end{minipage}

  \noindent\begin{minipage}{0.2\textwidth}% adapt widths of minipages to your needs
      \includegraphics[width=\linewidth]{figures/sdu_logo.pdf}
  \end{minipage}%
  \hfill%
  \begin{minipage}{0.8\textwidth}
    \begin{equation}
    \begin{align*}
      \sigma _{compression} = \sigma _{tension} &= \frac{M h_{CG}}{I_x} = \frac{4}{\pi r_1 ^3} M\\
      h_{CG} &= r_1 \\
      I_x = I_y &= \frac{\pi r_1 ^4}{4}
      \end{align*}
    \end{equation}
  \end{minipage}

  \noindent\begin{minipage}{0.2\textwidth}% adapt widths of minipages to your needs
      \includegraphics[width=\linewidth]{figures/sdu_logo.pdf}
  \end{minipage}%
  \hfill%
  \begin{minipage}{0.8\textwidth}
    \begin{equation}
    \begin{align*}
      \sigma _{compression} = \sigma _{tension} &= \frac{M h_{CG}}{I_x} = \frac{6 d}{(d^4 - k^4)}\\
      h_{CG} &= \frac{d}{2} \\
      I_x = I_y &= \frac{1}{12} (d^4 - k^4)
      \end{align*}
    \end{equation}
  \end{minipage}


% subsubsection profile_study (end)




% subsection limb_profile (end) 