%!TEX root = ../../report.tex
\chapter{State of the art} % (fold)
\label{cha:state_of_the_art}
This chapter contains a concise overview of some of the major, and more relevant for this thesis, advances in artificial legged locomotion in the past 60 years.
A brief prelude of the emergence and evolution of natural, limb-based terrestrial motion and more in detail bipedalism is presented.
It is followed by a short introduction of some of the history and results from the union between the engineering field and the study of motion and legged systems. 
From the first, ancient mobile artifacts and prosthesis to the current, most advanced devices in robotics.
It continues with an exposition of three of the most pertinent breakthroughs in modern technologies that allowed to reach the current state of the art in modern autonomous walking robots.
And it finishes with a quick look to the future of the bipedal locomotion.

%!TEX root = ../../../report.tex
\section{The study of bipedalism}
\label{sec:bipedalism}

The current human bipedal locomotion arises from the combination of a big variety of subsystems working in conjunction to achieve the intended gait generation according to the requirements of the situation.
The biomechanics of the limbs, consisting in bones, muscles and tendons under the control of the nervous system yields the adequate production of the different motion patterns in order to displace the body as energetically efficiently as possible.
This complex behavior is the result of 4 million years of an evolution \cite{bipedalism2} that started in primates and that has entailed both morphological and neurological changes in the human body since the first bipedal hominids to the current structure in Homo sapiens.
There exist several different theories about the reasons that originated and led to the adaption of this posture and bipedal motion.
Although different, most of them assume that most of them assume that the development of this structural and behavioral changes arose from a change in the environment in pre-hominids, for with bipedal behavior suddenly offered some kind of survival value \cite{bipedalism1}.
However, the genus Homo is not the only species that has evolved towards two-legged locomotion
Currently there are a few more species that have reached this method of displacement as a result of a natural selection process in which bipedalism offered the broadest set of advantages of the specie being the main ones listed below. 

\begin{itemize}
	\item Erect posture for a wider field of view and reach range.
	\item Free forelimbs, that could evolve towards specialized, non-locomotory applications such as object manipulation, combat, flight, etc.
	\item Faster displacement in certain species, although not generally.
\end{itemize}

Extensive research in the actuation and control structures involved in human bipedalism has been conducted from within the scientific fields of anthropology, biology, medicine, sport science and lately, several areas withing the engineering.
Its goal, as per definition of science, has been to reach a full understanding of what led to this behavior and the knowledge of how it functions, together with the discovery of ways in which it can be mimicked and improved, aiming at a more inexpensive and optimized locomotion.
This last fact has led to the belief that the next stage in the evolution of human locomotion will not come from nature as until nowadays, but from the hand of science and engineering, which has been lately depicted in literature and pop-culture as in \ref{fig:biped_evolution}.

\begin{figure}[h]
	\centering
	\includegraphics[width=0.8\textwidth]{figures/artificialhumans.jpg}
	\caption{Evolution of bipedalism (artistic depiction from \cite{human_evol_fig})}
	\label{fig:biped_evolution}
\end{figure}

\subsection{Bipedalism and engineering} % (fold)
\label{sub:bipedalism_and_engineering}
The discipline of engineering has been historically among the latest of the cited ones to start its contribution to the research and improvement of biped motion in a well defined manner, although its earliest contributions seem to date from ancient Egypt and Hinduism 
From the study of human bipedal motion described above, the insights emerged on its functioning together with the need to repair, improve or imitate its functionality led to the creation of new branches within the discipline of engineering.
The three most relevant ones for the present thesis are listed and introduced below.

\begin{enumerate}
	\item Prosthetics
	\item Orthotics
	\item Humanoid robots  **(find better name)
\end{enumerate}


\paragraph{Prosthetics} % (fold)
\label{par:prosthetics}

% paragraph prosthetics (end)

\paragraph{Orthotics} % (fold)
\label{par:orthotics}

% paragraph orthotics (end)

\paragraph{Humanoid robots} % (fold)
\label{par:humanoid_robots}

% paragraph humanoid_robots (end)

\begin{figure}[h]
	\centering
    \begin{subfigure}[b]{0.3\textwidth}
        \includegraphics[width=\textwidth]{figures/prosthetic_leg.png}
        \caption{Prosthetic leg, Colorado Springs Co.}
        \label{fig:prosthetic_leg}
    \end{subfigure}
    \centering
    \begin{subfigure}[b]{0.3\textwidth}
        \includegraphics[width=\textwidth]{figures/orthotic_leg.jpg}
        \caption{Knee brace, New Hope Co. model}
        \label{fig:orthotic_leg}
    \end{subfigure}
    \centering
    \begin{subfigure}[b]{0.3\textwidth}
        \includegraphics[width=\textwidth]{figures/robotic_leg.jpg}
        \caption{Robotic legs, COMAN robot \cite{coman}}
        \label{fig:robotic_leg}
    \end{subfigure}
\end{figure}
\todo{Rearrange/find other figures}


% subsection bipedalism_and_engineering (end)


%!TEX root= ../../../report.tex

\section{From the wheel to the leg} % (fold)
\label{sec:from_the_wheel_to_the_leg}
While the wheel is a young human invention meant to facilitate terrestrial motion, legged locomotion has been the result of millions of years of adaption of the species from aquatic to terrestrial fields.
Even though other forms of motion emerged during the same process, including limbless or rolling, the tetrapod and quadrupedal motion is the one that has proved the best performance in terrestrial displacement in relative velocity or jump length, for instance.

\subsection{The wheel in robotics} % (fold)
\label{sub:the_wheel_in_robotics}
Despite the facts above, it was the wheel the mean chosen to implement the ability to move around on the first mobile robots such as Walter's tortoises or the John Hopkins University robot Beast \cite{first_mobile_robot}, \cite{second_mobile_robot}.
This could be explained due to the fact that, as an artificial human creation, the modeling and control of the wheel could be fully mastered during its evolution process, easing its implementation in robotic platforms.
Some examples of this are shown in Figure \ref{fig:mobile_robots}, which depicts three different applications of wheeled robots.

The Rover "Spirit", in \ref{fig:mobile_rover}, is a good example of the possibilities and limits entailed by wheel-based locomotion.
It was successfully launched on Mars surface in 2004, which can be considered one of the most challenging and unpredictable environment a robot can be subject to, and it stayed active and operative until 2010.
However, the end of the mission came from the hand of the terrain. 
A very low-cohesion area of soil made the wheels of the device loose traction and eventually get trapped, preventing from recovering the control of the vehicle and bringing about the end of the mission.

Figure \ref{fig:mobile_mir} shows a MIR 100 robot, a state-of-the-art service robot designed for independent transportation and logistics in a dynamic and unpredictable environment..
As a service robot, it must fulfill strict safety standards and guarantee a secure interaction with users and other equipment while accomplishing its tasks.
Thanks to its wheels configuration, it can achieve an agile mobility, fast reaction capacity and a big load capacity.

The EMIEW 2, in \ref{fig:mobile_hitachi}, is an example of midpoint between wheeled and legged locomotion.
Although its motion relies on steering wheels, their placement at the end of two leg-like limbs provides with some of the advantages of bipedal motion, such as active suspension or impact absorption.

\begin{figure}[h]
\label{fig:mobile_robots}
	\centering
    \begin{subfigure}[b]{0.32\textwidth}
        \includegraphics[width=\textwidth]{figures/mobile_rover.pdf}
        \caption{Mars Rover Spirit}
        \label{fig:mobile_rover}
    \end{subfigure}
    \centering
    \begin{subfigure}[b]{0.32\textwidth}
        \includegraphics[width=\textwidth]{figures/mobile_mir.pdf}
        \caption{MIR 100 by MIR}
        \label{fig:mobile_mir}
    \end{subfigure}
    \centering
    \begin{subfigure}[b]{0.32\textwidth}
        \includegraphics[width=\textwidth]{figures/mobile_hitachi.pdf}
        \caption{EMIEW2, by Hitachi}
        \label{fig:mobile_hitachi}
    \end{subfigure}
\end{figure}

% subsection the_wheel_in_robotics (end)

\subsection{Achieving legged motion} % (fold)
\label{sub:legged_motion_in_robotics}
In spite of the great successes attained in the field of wheeled-robotics, possible thanks to the development in engines and the increasing computing capacity of processors, mobile robots based on wheels have not been able to achieve performances in motion comparable to the ones found in nature.
Specially in uneven terrains, velocity, agility and efficiency are still problems under study.
Besides, the absence of wheels in nature as a result of darwinian evolution \cite{dawkins} ultimately led to questioning if they are truly the best mean to overcome the challenges that displacements in uneven, unknown surfaces yield for robots.

\begin{figure}[h]
\label{fig:mobile_robots}
	\centering
    \begin{subfigure}[b]{0.32\textwidth}
        \includegraphics[width=\textwidth]{figures/starleth.jpg}
        \caption{STARLeth}
        \label{fig:mobile_rover}
    \end{subfigure}
    \centering
    \begin{subfigure}[b]{0.32\textwidth}
        \includegraphics[width=\textwidth]{figures/biped_kaist.jpg}
        \caption{Raptor}
        \label{fig:mobile_mir}
    \end{subfigure}
    \centering
    \begin{subfigure}[b]{0.32\textwidth}
        \includegraphics[width=\textwidth]{figures/biped_atlas.jpg}
        \caption{ATLAS}
        \label{fig:mobile_hitachi}
    \end{subfigure}
\end{figure}


% subsection legged_motion_in_robotics (end)





% section from_the_wheel_to_the_leg (end)
%!TEX root= ../../../report.tex

\section{Biomechanics meets Control} % (fold)
\label{sec:biomechanics_vs_control}
The first attempts to accomplish artificial walking machines in the mid-1950s were based on stiff structures and kinematic control.
There has been more than 50 years of development in several technology fields between them and the current soft, compliant and torque-controlled legged robots.
Among the main advances that gave rise to the evolution of the autonomous walking robots, three considered essential and of great relevance for this thesis are listened in \ref{list:leg_advances} and further discussed here.
Thanks to them, nowadays 

\begin{itemize}
\label{list:leg_advances}
	\item The realization of the importance that a dynamic and active control has on the walking behavior, as opposed to rigid, kinematics-based motion.
	\item The conception of the embodied AI. The influence of the body in the process of thinking.
	\item The improvements in sensors/actuators performance, material science, computing power and power sources.
\end{itemize}

\subsection{Dynamics of legged locomotion} % (fold)
\label{sub:dynamics_control}
Stiff position control for kinematic trajectory planning has lately demonstrated impressive capabilities like during the DARPA learning locomotion challenge.
However, the limitations of this kind of control such as the need of a very detailed knowledge  about the robot state vector and the environment have been also exposed.
Thus, it seems that the solution for robust and adaptive motion platforms has to go through the development of dynamics-based control models.
The first big steps taken in dynamic legged locomotion control took place in the Leg Laboratory, at the MIT Artificial Intelligence Laboratory by Marc Raibert and his team.
Their findings about the importance of active balance and the possibility of creating simple and generalizable control algorithms for complex dynamics legged systems \cite{mit_leg_lab1} pushed the development of walking robots to a new stage.


% subsection dynamics_control (end)

\subsection{The importance of embodiment in control} % (fold)
\label{sub:the_embodiment_}
%R. Peiffer paper
%Passive walkers
%Compliance
%Inspiration from nature
% subsection the_embodiment_ (end)

\subsection{Main advances in hardware} % (fold)
\label{sub:the_advances_in_hardware}
%First WAP robots bipedalism_history pag 28

% subsection the_advances_in_hardware (end)


% section biomechanics_vs_control (end)
%!TEX root= ../../../report.tex

\section{The next generation of bipeds} % (fold)
\label{sec:the_next_generation_of_bipeds}
It appears clear now that the evolution of bipedal locomotion has in the last decades suffered a wide-reaching change in its conditions with the entrance on the stage of robotics science and its research in walking machines.
It still stands difficult, however, to foresee the paths that the advancements in artificial bipedalism will take without the risk of getting lost in speculations.
Notwithstanding, the latest breakthroughs yield reasons to believe that the next generation of walking machines will resemble more and more the physiological designs developed by nature, although only to the extent that the state-of-the-art mechantronics allows.
This last fact entails that a complete imitation of living biological structures is yet out of the reach of engineering, thus making its goal not to just copy, but to understand the underlying principles that drive biology and transfer them to the new synthetic creations with the existing technology.



However, the latest advances in some of the youngest areas of the engineering seem to be close to bring about a change in the so far unaltered rules of the game.
The progresses in AI and in the design of genetic algorithms have arisen the possibility of not only striving for mimicking nature, but also replicating its fundamental processes, artificially accelerating and modifying them.
In the next decades, nature could stop being the only motor of evolution and from there, the possibilities are boundless.

% section the_next_generation_of_bipeds (end)
%%!TEX root = ../../report.tex
\chapter{State of the art} % (fold)
\label{cha:state_of_the_art}
%!TEX root = ../../../report.tex
\section{Motivation} % (fold)
\label{sec:sim_motivation}
What is a simulator, why to simulate

% section motivation (end)
%%!TEX root = ../../report.tex
\chapter{State of the art} % (fold)
\label{cha:state_of_the_art}
%!TEX root = ../../../report.tex
\section{Motivation} % (fold)
\label{sec:sim_motivation}
What is a simulator, why to simulate

% section motivation (end)
%%!TEX root = ../../report.tex
\chapter{State of the art} % (fold)
\label{cha:state_of_the_art}
\input{chapters/cha_state_of_the_art/Sections/sec_motivation}
%\input{chapters/cha_state_of_the_art/cha_state_of_the_art}
\input{chapters/cha_state_of_the_art/Sections/sec_theoretical_background}
%\section{Theoretical background}

%\section{Current research}
% chapter state_of_the_art (end)
%!TEX root = ../../../report.tex

\section{Theoretical background}
\label{sec_theoretical_background}



Difficulties to calculate compliance added by springs configuration for specific applications.
Possibility to change the springs configurations
Study of adaption of neural controllers to robot platforms whose mathematical model is hard to define due to added compliance.
Possibility of comparing controllers performance in two different platforms

%\section{Theoretical background}

%\section{Current research}
% chapter state_of_the_art (end)
%!TEX root = ../../../report.tex

\section{Theoretical background}
\label{sec_theoretical_background}



Difficulties to calculate compliance added by springs configuration for specific applications.
Possibility to change the springs configurations
Study of adaption of neural controllers to robot platforms whose mathematical model is hard to define due to added compliance.
Possibility of comparing controllers performance in two different platforms

%\section{Theoretical background}

%\section{Current research}
% chapter state_of_the_art (end)
%%!TEX root = ../../../report.tex

\section{Theoretical background}
\label{sec_theoretical_background}



Difficulties to calculate compliance added by springs configuration for specific applications.
Possibility to change the springs configurations
Study of adaption of neural controllers to robot platforms whose mathematical model is hard to define due to added compliance.
Possibility of comparing controllers performance in two different platforms

%\section{Theoretical background}



%But it has been in the last 30 years of the past century when the robotics field has started to focus on accomplishing bipedal locomotion, being the first model the WAM-1, built at Waseda University in 1967 \cite{}.
% Since this robot, the evolution of the platforms targeted at mimicking human locomotion has rapidly achieved great successes, which can be exemplified in robots as ASIMO, MABEL or BioBiped \cite{mabel}, \cite{biobiped}, %\cite{}.  
% However, even though the generation of stable walking gaits for biped platforms seems to have reached a next stage with the latest Atlas robot, the variances in kinematics, kinetics and control required for running and for the transition between the walking and running gaits still set out unsolved challenges.
% Besides, differences in energy consumption and balance control or trajectory generation still remain noticeable when comparing the performance of humanoid robots and humans \cite{h7}.

% In order to contribute to the study of stable walking and running gaits generation in bipedal robots, and aiming at providing a new tool to gain new insights on the transition between these gaits, this project offers a new robotic platform for continuing the research in these areas at the University of Southern Denmark.
% The current bipedal robot being utilized at the AI department at the Mærsk Mc-Kinney Møller Institute to benchmark neural controllers for locomotion is the DACbot walking robot, a next generation of RunBot \cite{runbot1} \cite{runbot2}.
% The limitations of this model when trying to approach human-like gait include among others the lack of actuated ankles or compliance in other joints besides the ankles. 
% Furthermore, its fixed structure and the way it was manufactured make modifications for different experiments or even repairs a hard task.
% A new bipedal platform offering a structure easy to modify and repair, low-cost, fully actuated and whose compliant components can be reconfigured, could be a valuable instrument for future studies.
% The RuBi robot, introduced here, wants to be presented as a mean to overcome the above mentioned restrictions and provide these new features. 
% Besides, the possibilities arisen from having two different study platforms in the department include being able to test neural controllers in different robots to observe their adaption, for instance.



%\section{Current research}
% chapter state_of_the_art (end)