%!TEX root= ../../../report.tex

\section{Biomechanics meets Control} % (fold)
\label{sec:biomechanics_vs_control}
The first attempts to accomplish artificial walking machines in the mid-1950s were based on stiff structures and kinematic control.
There has been more than 50 years of development in several technology fields between them and the current soft, compliant and torque-controlled legged robots.
Among the main advances that gave rise to the evolution of the autonomous walking robots, three considered essential and of great relevance for this thesis are listened in \ref{list:leg_advances} and further discussed here.
Thanks to them, nowadays 

\begin{itemize}
\label{list:leg_advances}
	\item The realization of the importance that a dynamic and active control has on the walking behavior, as opposed to rigid, kinematics-based motion.
	\item The conception of the embodied AI. The influence of the body in the process of thinking.
	\item The improvements in sensors/actuators performance, material science, computing power and power sources.
\end{itemize}

\subsection{Dynamics of legged locomotion} % (fold)
\label{sub:dynamics_control}
Stiff position control for kinematic trajectory planning has lately demonstrated impressive capabilities for instance during the last DARPA learning locomotion challenge.
However, the limitations of this kind of control such as the need of a very detailed knowledge  about both the robot state vector and the environment have been also exposed.
Thus, it seems clear that the solution for robust and adaptive motion platforms has to go through the development of dynamics-based control models.
The first big steps in dynamic legged locomotion control took place in the 90's in the Leg Laboratory, at the MIT Artificial Intelligence Laboratory carried out by Marc Raibert and his team.
Their findings about the importance of active balance for stability and the possibility of creating simple and generalizable control algorithms for complex dynamic legged systems \cite{mit_leg_lab1} pushed the development of walking robots to a new stage.
Besides, they were among the pioneers in the realization of the influence of the mechanical design together with the control in the generation of behaviors in locomotion.

Since this first approach, ...

% subsection dynamics_control (end)

\subsection{Embodied AI and locomotion} % (fold)
\label{sub:the_embodiment_}
The embodiment concept came up in the Artificial Intelligence field in the 1980's as a mean to explore the influence of the body and its characteristics in the process of thinking. 
In the very convenient words of Rolf Pfeifer and Fumiya Iida, "How does walking relate to thinking".
As introduce in \cite{pfeifer}, the original intention of artificial intelligence was not only to develop clever algorithms, but also to understand natural forms of intelligence that have more to do with the interaction with the real world.

The advent of this new approach in AI caused by itself an enormous boost in the locomotion field in robotics since researchers started working with mobile robots.
In the seek of the answer to the famous question "Why don’t plants have brains?", thehe believe that the answer could be their absence of movement led to the general use of the already available wheeled robots.



turn to nature as a source of inspiration and helped see the importance of the mechanics in the behavior


%R. Peiffer paper
%Passive walkers
%Compliance
%Inspiration from nature
% subsection the_embodiment_ (end)

\subsection{Main advances in hardware} % (fold)
\label{sub:the_advances_in_hardware}
%First WAP robots bipedalism_history pag 28

% subsection the_advances_in_hardware (end)


% section biomechanics_vs_control (end)