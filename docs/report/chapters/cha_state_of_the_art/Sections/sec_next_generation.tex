%!TEX root= ../../../report.tex

\section{The next generation of bipeds} % (fold)
\label{sec:the_next_generation_of_bipeds}
It appears clear now that the evolution of bipedal locomotion has in the last decades suffered a wide-reaching change in its conditions with the entrance on the stage of robotics science and its research in walking machines.
It still stands difficult, however, to foresee the paths that the advancements in artificial bipedalism will take without the risk of getting lost in speculations.
Notwithstanding, the latest breakthroughs yield reasons to believe that the next generation of walking machines will resemble more and more the physiological designs developed by nature, although only to the extent that the state-of-the-art mechantronics allows.
This last fact entails that a complete imitation of living biological structures is yet out of the reach of engineering, thus making its goal not to just copy, but to understand the underlying principles that drive biology and transfer them to the new synthetic creations with the existing technology.



However, the latest advances in some of the youngest areas of the engineering seem to be close to bring about a change in the so far unaltered rules of the game.
The progresses in AI and in the design of genetic algorithms have arisen the possibility of not only striving for mimicking nature, but also replicating its fundamental processes, artificially accelerating and modifying them.
In the next decades, nature could stop being the only motor of evolution and from there, the possibilities are boundless.

% section the_next_generation_of_bipeds (end)